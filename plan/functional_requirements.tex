\documentclass[10pt,letter]{article}
    % basic article document class

\usepackage{fullpage}
    % package that specifies normal margins
\usepackage{makecell}
\usepackage{setspace}
\usepackage{tabularx}


\doublespacing

\begin{document}
    % line of code telling latex that your document is beginning

\title{CS341 Shopping Cart Functional Requirements}

\author{Group 4: Sal Skare, Jack Englund, David Wissink, and John Collins}

\maketitle 
    % tells latex to follow your header (e.g., title, author) commands.

\section{About this document} This document describes the functional requirements for the 
software for an online e-commerce shopping cart software product.

\section{Functional Requirements}

\subsection{Users}

\begin{tabularx}{\textwidth}{l X}
    \it{Index:} & U.1 \\
    \it{Name:} & CreateUser \\
    \it{Purpose:} & To create a new user in the database. \\
    \it{Input parameters:} & Email, Password, Type of user \it{[customer, admin]}. \\
    \it{Action:} & Creates a new entry in the database for the user, hashing the password. \\
    \it{Output parameters:} & Success or failure message. \\
    \it{Exceptions:} & The user already exists, not all required fields were given, there 
    was a database error. \\
    \it{Remarks:} & It will be up to the front-end design to also log in the newly created 
    user if that functionality is desired. \\
    \it{Cross references:} & None. \\
    \hline
\end{tabularx}

\subsection{Promotions}

\begin{tabularx}{\textwidth}{l X}
    \it{Index:} & P.1 \\
    \it{Name:} & CreatePromotion \\
    \it{Purpose:} & To create a promotion for users to apply to items \\
    \it{Input parameters:} & Name, Discount Code, Discount Type \it{[bogo, percent]}, Percent, End date\\
    \it{Action:} & Creates a new promotion in the database.\\
    \it{Output parameters:} & Success or Failure message. \\
    \it{Exceptions:} & The discount code already exists, invalid end date \\
    \it{Remarks:} & The promotion should only be valid until the end date, the code can only be used once per user. \\
    \it{Cross references:} & None. \\
    \hline
\end{tabularx}

\begin{tabularx}{\textwidth}{l X}
    \it{Index:} & P.2 \\
    \it{Name:} & EndPromotion \\
    \it{Purpose:} & To forcefully end a promotion \\
    \it{Input parameters:} & Discount Code \it{[Identifier]}.\\
    \it{Action:} & Ends given promotion from database\\
    \it{Output parameters:} & Success or Failure message. \\
    \it{Exceptions:} & The discount code doesn't exist, the promotion is already ended \\
    \it{Remarks:} & Edit a field in the database, 'ongoing', to mark it as done. Otherwise could just change the end date to the current datetime.\\
    \it{Cross references:} & None. \\
    \hline
\end{tabularx}

\begin{tabularx}{\textwidth}{l X}
    \it{Index:} & P.3 \\
    \it{Name:} & GetPromotion \\
    \it{Purpose:} & Returns the type & percentage of the promotion. \\
    \it{Input parameters:} & Discount Code \it{[Identifier]}.\\
    \it{Action:} & Gets a promotion based on the discount code given.\\
    \it{Output parameters:} & Success or Failure message. \\
    \it{Exceptions:} & The discount code doesn't exist, the promotion is over. \\
    \it{Remarks:} & This should be used during ApplyPromotion\\
    \it{Cross references:} & None. \\
    \hline
\end{tabularx}

\begin{tabularx}{\textwidth}{l X}
    \it{Index:} & P.4 \\
    \it{Name:} & ApplyPromotion \\
    \it{Purpose:} & Applies the given promotion to a user's cart \\
    \it{Input parameters:} & Discount Code \it{[Identifier]}.\\
    \it{Action:} & Applies the discount percentage and type to the user's existing cart, modifies their total.\\
    \it{Output parameters:} & Success or Failure message. \\
    \it{Exceptions:} & The discount code doesn't exist, the promotion is over. \\
    \it{Remarks:} & This should be used when a user enters a discount code at checkout.\\
    \it{Cross references:} & None. \\
    \hline
\end{tabularx}

\end{document}
